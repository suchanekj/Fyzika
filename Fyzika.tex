\documentclass[titlepage]{book}

\usepackage{chngcntr}
\usepackage{fullpage}
\usepackage{tikz}
\usepackage{amsmath}
\usepackage{pgfplots}
\usepackage{listingsutf8}
\usepackage{graphicx}
\usepackage[main=czech]{babel}
\usepackage[utf8]{inputenc}
\usepackage{graphicx}
\usepackage{amsfonts}
\usepackage{hyperref}

\counterwithin*{equation}{paragraph}

\lstset{literate=
  {á}{{\'a}}1 {é}{{\'e}}1 {í}{{\'i}}1 {ó}{{\'o}}1 {ú}{{\'u}}1 {ý}{{\'y}}1
  {Á}{{\'A}}1 {É}{{\'E}}1 {Í}{{\'I}}1 {Ó}{{\'O}}1 {Ú}{{\'U}}1 {Ý}{{\'Y}}1
  {ě}{{\v{e}}}1 {š}{{\v{s}}}1 {č}{{\v{c}}}1 {ř}{{\v{r}}}1 {ž}{{\v{z}}}1
  {Ě}{{\v{e}}}1 {Š}{{\v{S}}}1 {Č}{{\v{C}}}1 {Ř}{{\v{R}}}1 {Ž}{{\v{Z}}}1
}

\title{Maturitní příprava k fyzice}
\author{Jakub Suchánek}
\date{ }

\setcounter{tocdepth}{1}
\renewcommand*\contentsname{Obsah}
 
\begin{document}

\maketitle
\tableofcontents{}

\part{Pokročilejší témata}
\chapter{Matematika}
\section{Vektory}
\subsection{Notace}
$\boldsymbol x$ znamená vektor x, tedy tučné symboly jsou vektory.
\subsection{Pravidlo pravé ruky}
$\boldsymbol a \times \boldsymbol b = \boldsymbol c$\\
Natažený ukazováček ve směru $\boldsymbol a$, prostředníček zahnutý do pravého úhlu ve směru $\boldsymbol b$, palec je do strany a ukazuje směr $\boldsymbol c$. 
\section{Kombinatorika}
První dva známé fakty o kombinačních číslech:\\
\begin{enumerate}
\item Kombinační číslo $\binom{n}{k}$ se dá spočítat paskalovým trojúhelníkem jako k-té číslo na n-tém řádku (Paskalův trojúhelník je indexovaný od 0).
\item Kombinační číslo $\binom{n}{k}$ se dá spočítat jako $\frac{n!}{k!(n-k)!}$.
\end{enumerate}
První se zamyslíme nad 2. $\frac{n!}{(n-k)!}$ nám říká kolika způsoby můžeme vybrat k prvků, protože vždy když vybíráme další člen, máme o jednu možnost méně a když už máme k prvků tak skončíme. Část $\frac{1}{k!}$ je tam kvůli tomu, že nás nezajímá v kterém pořadí jsme těch k členů získali.\\
Paskalův trojúhelník je mnohem zajímavější. Pokud se podíváme na nějaké číslo vede do něj z počátku mnoho cest složených z kroků doleva a doprava. Krok doleva si představíme jako není tam a krok doprava jako je tam. Proto n-tý řádek - rozhodovali jsme o n členech zda tam budou - a k-tý prvek na řádku - k-krát jsme řekli je tam. Takhle spočítáme kolik do k-tého čísla n-tého řádku vede cest, odpovídajících možným vybraným kombinacím, a to je právě kombinační číslo. Také si všimneme že součet každého řádku je $2^n$.
\subsection{Gausova křivka}
Ve fyzice se často u náhodných veličin objevuje normální rozdělení (gausova křivka) a důvod pro to je, že normálové rozložení odpovídá přibližně tomu, že na osu x dáme $k$ a na osu y $\frac{\binom{n}{k}}{2^n}$, chyba je menší pro velké n. Normálové rozložení tedy dává rozložení pravděpodobností a pokud máme mnoho členů, reálné rozložení je mu velmi blízko (molekuly ve vzduchu, lidská výška - určována n různými geny).
\section{Komplexní čísla}
\subsection{Sčítání}
Sčítání komplexních čísel provádíme po složkách, tedy stejně jako u vektorů.
\subsection{Násobení}
Násobení můžeme také provádět po složkách, ale mnohem zajímavější je vynásobit velikosti a sečíst úhly. Důkaz:\\
Heldáme $(a + bi)(c + di)$, najdeme $x,y,\alpha, \beta$ takové, že $x \cos\alpha = a, x \sin\alpha = b, y \cos\beta = c, y \sin\beta = d$. Potom:
\begin{multline}
(x \cos\alpha + x \sin\alpha i)(y \cos\beta + y \sin\beta i) = xy\cos\alpha\cos\beta - xy\sin\alpha\sin\beta + xy\sin\alpha\cos\beta i + xy\cos\alpha\sin\beta i \\
= xy\cos(\alpha + \beta) + xy\sin(\alpha + \beta) i
\end{multline}
\section{Trocha kalkulu}
\subsection{Literatura}
Kalkulus tu celý vysvětlovat nebudu, jen pár symbolů co budu dále používat, pokud kalkulus neumíte a nelíbí se vám školní učebnice, zde je trocha literatury:\\
Americká kniha, dobře se z ní učí, Peterson's Master the AP Calculus: \url{https://shamokinmath.wikispaces.com/file/view/Peterson's+Master+AP+Calculus.pdf}\\
Pokud chcete i důkazy a mnoho teorie, Matfyzácká skripta z matematické analýzy: \url{http://www.karlin.mff.cuni.cz/~pick/analyza-pro-studenty.pdf}\\
Derivace ve fyzice, FO: \url{http://fyzikalniolympiada.cz/texty/matematika/difpoc.pdf}\\
Integrály ve fyzice, FO: \url{http://fyzikalniolympiada.cz/texty/matematika/intpoc.pdf}\\
Diferenciální rovnice ve fyzice, FO: \url{http://fyzikalniolympiada.cz/texty/matematika/difro.pdf}
\subsection{Notace}
$\int \boldsymbol f(\boldsymbol l) \cdot d\boldsymbol l$ znamená, že máme integrovat podle křivky $l$ a zároveň vynásobit kosínem úhlu $\boldsymbol f(\boldsymbol l)$ vůči křivce. Podobně tam může být vektorové křížové násoben nebo místo křivky můžeme integrovat podle plochy (potom bude náobení (křížové či skalární) vůči normálovému vektoru. $\oint \boldsymbol f(\boldsymbol l) \cdot d\boldsymbol l$ znamená, že křivka je uzavřená (například kružnice), podobně pro plochu (například sféra (povrch koule)).
\subsection{Taylorův polynom}
Používá se pro aproximaci polynomem v okolí zvoleného bodu, když je funkce složitá ale jsme schopni určit hodnotu několika derivací ve zvoleném bodě. Je definovaný vzorcem (pro funkci $f(x)$ z bodu $t$):
\begin{equation}
T_n^{f,a}(x) = \frac{f(t)(x-t)^0}{0!} + \frac{f'(t)(x-t)^1}{1!} + \frac{f''(t)(x-t)^2}{2!} + \frac{f'''(t)(x-t)^3}{3!} + ... + \frac{f^{(n)}(t)(x-t)^n}{n!}
\end{equation}
\section{Koule se občas rovná bod}
Pokud máme rovnici tvaru $f(r) = \frac{k}{r^2}$ jako třeba u gravitačního či elektrostatického pole a jim odpovídajícím silám, a dané pole vychází z koule, kterou můžeme rozdělit na slupky o konstantní hustotě (planety, dutá koule, koule...), potom můžeme daný objekt nahradit hmotným bodem. Důkaz: \url{http://hyperphysics.phy-astr.gsu.edu/hbase/Mechanics/sphshell.html}\\
\section{Tok a uzavřené plochy}\label{sec:tok}
Pokud pracujeme s takovouto funkcí ($f(r) = \frac{k}{r^2}$) udávající pole, a zkoumáme objekt uvnitř uzavřené plochy, potom tok plochou nezávisí vůbec na tom jak ta plocha vypadá. Podobně to je u jiných toků, třeba tok kapaliny. Tok, označovaný $\Phi$, je v případě elektrického pole vyjádřený jako $\int \boldsymbol E \cdot d\boldsymbol s$, kde E je intenzita elektrického pole, jindy to může být množství kapaliny protékající plochou, tedy ve vzorci se intenzita pole nahradí za rychlostní vektor. Představme si sféru těsně obalující zkoumané těleso. Když ji budeme rozšiřovat tak se nebude měnit tok (roste plocha a klesá  intenzita/rychlost) a i když není uzavřená plocha sférou, efektivní plochu má stejnou (proto je tam skalární součin).
\section{Síla a sféry}
Pokud pracujeme se silou udanou funkcí ($f(r) = \frac{k}{r^2}$), a zkoumáme objekt uvnitř sféry, nepůsobí na néj od ní žádná celková síla. Narozdíl od předchozího případu nemůžeme využít efektivní plochu, protože pokud je plocha v daném bodě pod větším úhlem má tam více hmoty/náboje. Pokud se ale podíváme na dva body na kouli v opačném směru, budou tečné plochy v těch bodech svírat stejný úhel vůči vzdálenosti mezi tělesem a daným bodem. Pokud toto platí tak nám to stačí, jelikož s rostoucí vzdáleností roste i plocha. Nyní ješté dokázat to tvrzení. Vezmeme si libovolný řez sférou, který obsahuje oba body (a tedy i těleso). Libovolný řez sférou bude kružnice a pokud pro všechny řezy budou tečny v bodech svírat stejný úhel k vzdálenostem, pak to bude platit i pro plochy. Tečné body a průsečík tečen vždy tvoří rovnoramený trojúhelník, a můžeme si všimnout, že jelikož jsou body od tělesa v opačných směrech, jsou i s tělesem na přímce, a tedy na základně rovnorameného trojúhelníku.
\chapter{Elektřina a magnetismus}
\section{Elektrické pole}
\subsection{Vodiče a nevodiče}
Elektrické pole v dokonalém vodiči je vždy nulové, proto všechen náboj ve vodiči je na povrchu (viz. \ref{sec:tok}). Pro nevodiče to neplatí.
\subsection{Gaussův zákon}
Gauss odvodil myšlenku \ref{sec:tok} pro elektrické pole:\\
\begin{equation}\label{eq:random}
\Phi_E = \int_S E \cdot dS = \frac{Q}{\varepsilon}
\end{equation}
\section{Náhodné procházky elektronů}
\subsection{Náhodné procházky po grafech}
Mějme diskrétní graf $G(V,E)$. Náhodná procházka bude probíhat tak, že pokud se nacházíme v nějakém vrcholu tak si rovnoměrně náhodně vybereme jednoho ze sousedů do kterého se vydáme. Pravděpodobnostní rozložení $P$ říká pro každý vrchol jaká je pravděpodobnost, že se na něm nacházíme. Každý graf má právě jedno stabilní pravděpodbnostní rozložení, tedy rozložení na kterém se časem ustálí. V tom bude platit:\\
\begin{equation}
\forall v \in V: \sum_{u: (v,u) \in E} P_u - P_v = 0
\end{equation}. Zavedeme do grafu několik bodů, kde bude pevně nastavená pravděpodobnost (musíme ji nastavit tak, aby na nich v průměru stále byla pravděpodbnost výskytu $\frac{1}{n}$). I pro takový grav bude stabilní rozložení existovat.
\subsection{Aplikace na elektrický proud}
Představíme si, že na každé hraně je jednotkový rezistor. Všimneme si, že rovnice \ref{random} odpovídá Kirchhoffovu zákonu o proudu. 
\subsection{Literatura}
Podrobně s důkazy to můžete nalézt v této knize \\ \url{https://rajsain.files.wordpress.com/2013/11/randomized-algorithms-motwani-and-raghavan.pdf}
\section{Maxwellovy rovnice}
\subsection{1}
\chapter{Vesmír}
\section{Speciální teorie relativity}
\section{Obecná teorie relativity}
\section{Gravitační vlny}

\part{Podle otázek}
\setcounter{chapter}{0}

\chapter{Kinematika hmotného bodu}
Popisuje pohyb těles, ale nezabývá se příčinami pohybu.
\section{Hmotný bod}
Bezrozměrné těleso s přiřazenou hmotností. Zanemedbává tedy rozměry a nezanedbává hmotnost.
\section{Vztažná soustava}
Jelikož neexistuje éter, tedy nějaká nehybná substance, ke které můžeme vztáhnout pohyb, musíme si zvolit skupinu těles a prohlásit je za nehybné.
\subsection{Inerciální vztažná soustava}
Inerciální vztažná soustava je taková vztažná soustava, kde platí 1. Newtonův zákon, tedy těleso se pohybuje rovnoměrně přímočaře právě tehdy, když výslednice sil na něj působících je nulová.
\subsection{Neinerciální vztažná soustava}
Neinerciální vztažná soustava je taková vztažná soustava, kde neplatí 1. Newtonův zákon.
\section{Relativnost klidu a pohybu}
Neexistuje éter, takže klid a pohyb se musí určovat podle vztažné soustavy.
\section{Kinematické veličiny}
Dráha $s$\\
\begin{equation}
\boldsymbol s = \int \boldsymbol v dt
\end{equation}
Rychlost $v$\\
\begin{equation}
\boldsymbol v \equiv \frac{d\boldsymbol s}{dt} = \int \boldsymbol a dt
\end{equation}
Zrychlení $a$\\
\begin{equation}
\boldsymbol a \equiv \frac{d\boldsymbol v}{dt}
\end{equation}
Úhlová dráha $\theta$\\
\begin{equation}
\boldsymbol \theta \equiv \frac{\boldsymbol s \times \boldsymbol r}{r^2} = \int \omega dt
\end{equation}
Úhlová rychlost $\omega$\\
\begin{equation}
\boldsymbol \omega \equiv \frac{d\boldsymbol \theta}{dt} = \int \boldsymbol \alpha dt = \frac{\boldsymbol v \times \boldsymbol r}{r^2}
\end{equation}
Úhlové zrychlení $\alpha$\\
\begin{equation}
\boldsymbol \alpha \equiv \frac{d\boldsymbol \omega}{dt} = \frac{\boldsymbol a \times \boldsymbol r}{r^2}
\end{equation}
Dostředivé zrychlení $a_{do}$\\
\begin{equation}
a_{do} \equiv -\omega^2 \boldsymbol r
\end{equation}
Perioda $T$\\
Frekvence $f$\\
\begin{equation}
f \equiv \frac{1}{T}
\end{equation}
\section{Jednotky a vztahy pro rovnoměrný a rovnoměrně zrychlený pohyb hmotného bodu přímočarý i po kružnici}
\subsection{Pohyb přímočarý}
\begin{align}
s &= s_0+v_0t+\frac{1}{2}at^2                            &  \Big| \quad &m\\
v &= v_0+at = \sqrt{v_0^2 + 2a(s-s_0)}                   &  \Big| \quad &\frac{m}{s}\\
v_{prum} &= \frac{\Delta s}{t}                           &  \Big| \quad &\frac{m}{s}\\
a &= \frac{\Delta v}{t} = \frac{2\Delta s}{t^2}          &  \Big| \quad &\frac{m}{s^2}
\end{align}
\subsection{Pohyb po kružnici}
\begin{align}
\theta &= \theta_0 + \omega_0t+\frac{1}{2}\alpha t^2                                     &  \Big| \quad &rad\\
\omega &= \omega_0 + \alpha t = 2 \pi f  = \sqrt{\omega_0^2 + 2\alpha(\theta-\theta_0)}  &  \Big| \quad &\frac{rad}{s}\\
\omega_{prum} &= \frac{\Delta \theta}{t}                                                 &  \Big| \quad &\frac{rad}{s}\\
\alpha &= \frac{\Delta \omega}{t} = \frac{2\Delta \theta}{t^2}                           &  \Big| \quad &\frac{rad}{s^2}\\
a_{do} &= \omega^2r                                                                      &  \Big| \quad &\frac{m}{s^2}\\
v &= \omega r                                                                            &  \Big| \quad &\frac{m}{s}\\
T &= \frac{1}{f} = \frac{2\pi}{\omega}                                                   &  \Big| \quad &s\\
f &= \frac{1}{T} = \frac{\omega}{2\pi}                                                   &  \Big| \quad &Hz
\end{align}
\section{Grafy}
\chapter{Dynamika křivočarých a přímočarých pohybů}
\section{Newtonovy pohybové zákony}
\subsection{1. - zákon setrvačnosti}
Těleso setrvává v rovnoměrném přímočarém pohybu, je-li výslednice vnějších sil působících na těleso nulová.
\subsection{2. - zákon síly}
\begin{equation}
\boldsymbol F_{vys} = m\boldsymbol a
\end{equation}
\subsection{3. - zákon akce a reakce}
Působí li těleso silou, je na něj působeno stejnou silou v opačném směru.
\section{Zákon zachování hybnosti (ZZH)}
Hybnost uzavřené soustavy se zachovává.
\section{Souvislost pohybových zákonů s volbou vztažné soustavy}
Inerciální vztažná soustava je definovaná tak, že musí platit 1. Newtonův zákon, a tedy i ZZH. V neinerciální soustavě tedy všechny pohybové zákony neplatí.
\section{Podmínky platnosti zákonů v klasické mechanice}
Inerciální vztažná soustava.\\
Rychlosti nejsou relativistické ($v << c$), tedy můžeme aproximovat\\
\begin{equation}
\sqrt{1-\frac{v^2}{c^2}} = 1
\end{equation}
\chapter{Druhy energií a jejich proměny}
\section{Mechanická energie}
\subsection{Potenciální/polohová energie}
Předpokládá homogení gravitační pole, může být tedy použita při malých rozměrech (maximálně v řádu kilometrů) v blízkosti země.\\
\begin{equation}
E_p = mgh
\end{equation}
Vychází to z obecné gravitační potenciální energie
\begin{equation}
E_p = -G\frac{Mm}{r}
\end{equation}
Kde se předpokládá $g$ konstanta a to:
\begin{equation}
g = G\frac{M}{r^2}
\end{equation}
\subsection{Kinitecká/pohybová energie}
\begin{equation}
E_k = \frac{1}{2}mv^2
\end{equation}
Odvození z $E = mc^2$
\begin{align}
E &= \frac{m_0}{\sqrt{1 - \frac{v^2}{c^2}}}c^2\\
E_k &= m_0 \Big(\frac{1}{\sqrt{1 - \frac{v^2}{c^2}}} - 1\Big)c^2
\end{align}
Napíšeme si Tailorův polynom pro v = 0:
\begin{equation}
T^{E_k,0} = \frac{1}{2}m_0v^2 + \frac{3}{8}m_0\frac{v^4}{c^2} + \frac{5}{16}m_0\frac{v^6}{c^4} + O(v^8)
\end{equation}
Všechny členy kromě prvního můžeme pro malé $v$ zanedbat.
\section{Mechanická práce}
\begin{equation}
W = \int \boldsymbol F \cdot d\boldsymbol s
\end{equation}
\section{Mechanická práce}
\begin{equation}
P = \frac{dW}{dt} = \boldsymbol F \cdot \boldsymbol v
\end{equation}
\subsection{Účinnost}
\begin{equation}
\eta = \frac{P}{P_0}
\end{equation}
\section{Teplo}
\begin{equation}
Q = mc\Delta t
\end{equation}
\subsection{Skupenské teplo}
\begin{equation}
Q = m l
\end{equation}
\section{Přenos vnitřní energie}
První termodinamický zákon:
\begin{equation}
\Delta U = W + Q
\end{equation}
\section{Jouleovo teplo}
\begin{equation}
P = UI = \frac{U^2}{R} = I^2R
\end{equation}
\section{Energie magnetického pole cívky}
\begin{equation}
E_m = \frac{1}{2} L I^2
\end{equation}
\section{Energie elektrického pole kondenzátoru}
\begin{equation}
E_e = \frac{1}{2} C U^2
\end{equation}
\section{Přeměny energie v oscilátorech}
\subsection{Mechanický oscilátor}
\begin{align}
E &= E_k + E_p\\
E_K &= \frac{1}{2}mv^2 = \frac{1}{2} m v_{max}^2 \sin^2(\omega t + \varphi)\\
E_p &= E_{k \ max} - E_k = \frac{1}{2}mv^2 = \frac{1}{2} m v_{max}^2 \cos^2(\omega t + \varphi)\\
E_p &= mgh + \frac{1}{2} - k \Delta l^2\\
&= E_{p0} + mg y_{max} (\cos(\omega t + \varphi) + 1) + k y_{max}^2 (\cos(\omega t + \varphi) - 1)^2
\end{align}
\subsection{LC oscilátor}
\begin{align}
E_m &= \frac{1}{2} L I^2 = \frac{1}{2} L I_{max}^2 \sin^2(\omega t + \varphi)
E_e &= \frac{1}{2} C U^2 = \frac{1}{2} C U_{max}^2 \cos^2(\omega t + \varphi)
\end{align}
\section{Zákony zachování energie}
\subsection{Zákon zachování mechanické energie}
\begin{equation}
E = E_k + E_p = konst.
\end{equation}
ZZME platí pouze v klasické mechanice za předpokladu, že všechny srážky předmětů jsou dokonale pružné.
\subsection{Zákon zachování energie}
Tento zákon už platí obecně, počítá totiž se všemi energiemi - energie pole (potenciální energie, vazebná energie), kinetické energie (mechanická, teplo)...
\chapter{Mechanika tuhého tělesa}
Nelze zanedbat rozměry.\\
Zanedbáváme veškeré deformační účinky.
\section{Posuvný a otáčivý pohyb tuhého tělesa}
Pro každou sílu počítáme zvlášť sílu přenesenou do těžiště a moment síly z těžiště. Obě vektorově sčítáme. Výslednice sil dává (po vydělení hmotností) zrychlení a výslednice momentů rotační zrychlení.
\section{Výsledek působení sil na těleso}
Translační:
\begin{align}
\boldsymbol F &= \Sigma \boldsymbol F_i\\
\boldsymbol a &= \frac{\boldsymbol F}{m}
\end{align}
Rotační:
\begin{equation}
\boldsymbol M = \int \boldsymbol F \times d\boldsymbol r
\alpha = \frac{M}{m}
\end{equation}
\section{Dokonale tuhé těleso}
Rozměry se pod působením sil nemění.
\section{Momentová věta}
Pokud vektorový součet momentů na těleso otáčivé kolem pevné osy je nulový, má těleso nulové rotační zrychlení.
\section{Jednoduché stroje}
Páka, nakloněná rovina
\section{Těžiště}
\begin{equation}
\boldsymbol x_T = \frac{\int \boldsymbol r dm}{m_{celk}}
\end{equation}
\chapter{Mechanika kapalin a plynů}

\section{Struktura tekutin}

\section{Zákony statiky a dynamiky tekutin}

\chapter{Gravitační pole, pohyby v tomto poli}
\section{Všeobecný gravitační zákon}
\section{Veličiny gravitačního pole}
\section{Gravitační a tíhové pole}
\section{Pohyby v radiálním a homogenním poli}
\chapter{Elektrostatické pole}
\section{Vlastnosti elektrického náboje}
\section{veličiny popisující elektrické pole}
\section{Znázornění pole}
\section{Vodič a izolant v elektrickém poli}
\section{kondenzátory}
\chapter{Základní poznatky molekulové fyziky}
\section{Kinetická teorie látek}
\section{Vzájemné působení částic a jejich energie v různých skupenctvích}
\section{Statistický přístup}
\chapter{Základy termodynamiky}
\section{Vnitřní energie soustavy}
\section{Teplo, teplota}
\section{Kalorimetr}
\section{Přenos vnitřní energie}
\section{Tepelné motory}
\section{Termodynamické zákony}
\chapter{Struktura a vlastnosti plynů}
\section{Ideální plyn}
\section{Stavové veličiny}
\section{Stavové rovnice}
\section{Děje v plynech}
\section{Grafy}
\chapter{Struktura a vlastnosti pevných látek}
\section{Krystalická a amorfní látka}
\section{Ideální a kutečný krystal}
\section{Vakance}
\section{Deformace}
\section{Hookův zákon}
\section{Teplotní roztažnost}
\chapter{Elektrický proud v látce}
\section{Podmínky vedení proudu}
\section{Veličiny proud, napětí, odpor}
\section{Vedení proudu v kapalinách, plynech a polovodičích}
\chapter{Polovodiče}
\section{Příměsová a vlastní vodivost}
\section{Vysvětlení PN přechodu}
\section{Polovodičové součástky}
\chapter{Stejnosměrný proud}
\section{Elektrický proud v kovech}
\section{Ohmův zákon}
\section{Lineární vodiče}
\section{Zdroje}
\section{Kirchhoffovy zákony}
\section{Zapojování rezistorů}
\section{Práce a výkon elektrického proudu}
\chapter{Magnetické pole}
\section{Pole permanentního magnetu}
\section{Pole vodiče s proudem}
\section{Rozdělení magnetických látek}
\section{Působení magnetického pole na vodič a částice s nábojem}
\chapter{Nestacionární magnetické pole}
\section{Elektromagnetická indukce}
\section{Magnetický indukční krok}
\section{Fradayův zákon}
\section{Lenzův zákon}
\section{Užití elektromagnetické indukce}
\chapter{Mechanické a elektrické kmity}
\section{Nestacionární děje s periodickým průběhem}
\section{Typy oscilátoru}
\section{Veličiny kmitavého děje}
\section{Skláďání kmitů}
\section{Nucené kmitání}
\section{Rezonance oscilátorů}
\section{Přeměny energie v oscilátorech}
\chapter{Střídavý proud}
\section{Veličiny střídavého proudu}
\section{Jednoduché obvody střídavého proudu}
\section{Složené obvody střídavého proudu}
\section{Výkon střídavého proudu}
\section{Generátor}
\section{Spotřebiče střídavého proudu}
\chapter{Mechanické vlnění}
\section{Vznik}
\section{Šíření vlnění}
\section{Rovnice vlnění}
\section{Odraz}
\section{Lom}
\section{Ohyb a stín vlnění}
\section{Vlastnosti zvuku}
\chapter{Elektromagnetické vlnění}
\section{Vznik}
\section{Charakteristika elektromagnetického vlnění}
\section{Šíření vlnění}
\section{Přenos signálu elektromagnetickým vlněním}
\chapter{Vlnové vlastnosti světla}
\section{Světlo jako druh vlnění}
\section{Složené nebo monochromatické světlo}
\section{Rychlost světla v různých prostředích}
$c = 3 \cdot 10^8 \frac{m}{s}$\\

\section{Jevy, které potvrzují vlnovou teorii světla}
\section{Disperze, interference, difrakce}
\section{Odraz, lom a polarizace}
\chapter{Optické zobrazení a optické soustavy}
\section{Geometrická optika}
\section{Čočky a zrcadla}
\section{Konstrukce obrazu}
\section{Zobrazovací rovnice}
\section{Oko}
\section{Optické přístroje}
\chapter{Kvantová fyzika}
\section{Fotoelektrický jev}
\section{Planckova teorie}
\section{Foton}
\section{Comptonův jev}
\section{Dualismus vln a částic}
\section{De Broglieho vlny}
\chapter{Atomová a jaderná fyzika}
\section{Modely atomu}
\section{Periodická soustava prvků}
\section{Elektronový obal z hlediska kvantových částic}
\section{Laser}
\section{Rentgenové záření}
\section{Atomové jádro}
\section{Radioaktivita}
\chapter{Vesmír}
\section{Sluneční soustava}
\section{Keplerovy zákony pohybu planet}
\section{Teorie velkého třesku a rozpínání vesmíru}
\section{Speciální teorie relativity}
\section{Obecná teorie relativity}
\end{document}